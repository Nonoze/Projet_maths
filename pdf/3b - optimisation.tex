\subsubsection*{(iiib) Trouver une approximation de la solution optimale par une méthode d'optimisation numérique.}

\vspace{1\baselineskip}

Tout d'abord, voici les valeurs obtenues empiriquement pour \( m(\ell) \) avec \( \ell \in \llbracket 1, 213 \rrbracket \):

\begin{center}
	\includegraphics[width=0.7\textwidth]{../data/graph_f_bornes.png}
\end{center}


Nous avons émis l'hypothèse que la fonction \(m(\ell)\) valait approximativement \( k \sqrt{\ell} \) pour une certaine constante \( k \). Nous pouvons donc construire ce graphique des valeurs de \( \frac{m(\ell)}{\sqrt{\ell}} \):
\begin{center}
	\includegraphics[width=0.7\textwidth]{../data/graph_segment_over_sqrt_length.png}
\end{center}

Toutes les valeurs sont effectivement entre $\sqrt{2}\approx 1.414$ et $2$, ce qui confirme nos bornes. Cependant, nous pouvons remarquer que les valeurs semblent converger vers une valeur entre $1.7$ et $1.8$ lorsque \( \ell \) devient grand.

Même avec l'aide de toutes ces approximations, il est difficile de prouver rigoureusement une quelconque convergeance. Nous savons donc que \( m(\ell) \) est proportionnel à \( \sqrt{\ell} \), mais nous ne pouvons pas affirmer avec certitude que la limite de \( \frac{m(\ell)}{\sqrt{\ell}} \) existe.