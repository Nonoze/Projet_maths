\subsubsection*{(iiib) Trouver une approximation de la solution optimale par une méthode d'optimisation numérique.}

\vspace{1\baselineskip}

Tout d'abord, nous pouvons placer nos valeurs connues sur un graphique:

\begin{center}
	\includegraphics[width=0.7\textwidth]{../data/graph_f_bornes.png}
\end{center}


Nous avons émis l'hypothèse que la fonction \(f(x)\) valait approximativement \( k \sqrt{x} \) pour une certaine constante \( k \). Nous pouvons donc construire ce graphique des valeurs de \( \frac{f(x)}{\sqrt{x}} \):
\begin{center}
	\includegraphics[width=0.7\textwidth]{../data/graph_segment_over_sqrt_length.png}
\end{center}

Toutes les valeurs sont effectivement entre $\sqrt{2}\approx 1.414$ et $2$, ce qui confirme nos bornes. Cependant, nous pouvons remarquer que les valeurs semblent converger vers une valeur entre $1.7$ et $1.8$ lorsque \( x \) devient grand.

Même avec l'aide de toutes ces approximations, il est difficile de prouver rigoureusement une quelconque convergeance. Nous savons donc que \( f(x) \) est proportionnel à \( \sqrt{x} \), mais nous ne pouvons pas affirmer avec certitude que la limite de \( \frac{f(x)}{\sqrt{x}} \) existe.