\subsubsection*{(iiib) Trouver une approximation de la solution optimale par une méthode d'optimisation numérique.}

\vspace{1\baselineskip}

Tout d'abord, voici les valeurs obtenues empiriquement pour \( m(\ell) \) avec \( \ell \in \llbracket 1, 213 \rrbracket \):

\begin{center}
	\includegraphics[width=0.7\textwidth]{../data/graph_f_bornes.png}
\end{center}


Nous avons émis l'hypothèse que la fonction \(m(\ell)\) valait approximativement \( k \sqrt{\ell} \) pour une certaine constante \( k \). Nous pouvons donc construire ce graphique des valeurs de \( \frac{m(\ell)}{\sqrt{\ell}} \):
\begin{center}
	\includegraphics[width=0.7\textwidth]{../data/graph_segment_over_sqrt_length.png}
\end{center}

Toutes les valeurs sont effectivement entre $\sqrt{2}\approx 1.414$ et $2$, ce qui confirme nos bornes. Cependant, nous pouvons remarquer que les valeurs semblent converger vers une valeur entre $1.7$ et $1.8$ lorsque \( \ell \) devient grand.

\newpage

En cherchant une fonction du type \( m(\ell) = \sqrt{a\ell + b} \), nous sommes parvenus à la fonction suivante: $m(\ell) \approx \sqrt{3\ell + 2.05}$. Voici le graphique de cette fonction par rapport aux valeurs empiriques de \( m(\ell) \):
\begin{center}
	\includegraphics[width=0.7\textwidth]{../data/graph_f_fit.png}
\end{center}

Remarque 1: la fonction $\sqrt{3\ell+2.25}$ fonctionne avec moins de résidus, mais avec plus de valeurs de la forme $k+\frac{1}{2}$ où $k \in \mathbb{N}$. En arrondissant à l'entier le plus proche, pour la première fonction, on obtient 11 erreurs, et pour la 2ème, 8 erreurs.

Remarque 2: La fonction $\sqrt{3\ell+1.77}$ fait aussi partie des possibilités.