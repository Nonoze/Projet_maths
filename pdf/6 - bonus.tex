\subsubsection*{(bonus) Autres découvertes}

Au-delà des bornes établies précédemment pour $m(\ell)$, plusieurs autres observations intéressantes ont été faites lors de l'étude des règles à mesurer.

\vspace{1\baselineskip}

\textbf{1. Lien entre règle circulaire et jeu de société dobble:} Le jeu de société dobble peut être résolu en utilisant une règle circulaire parfaite. En effet, on associe chaque symbole du jeu à une graduation (utilisée ou non) sur une règle circulaire. On prend ensuite les graduations utilisées pour construire la règle parfaite sur la première carte, puis on décale les graduations de 1 pour obtenir la deuxième carte, et ainsi de suite jusqu'à revenir à la carte initiale. Chaque carte du jeu correspond ainsi à une position de la règle circulaire, garantissant que chaque paire de cartes partage exactement un symbole commun. La propriété est respectée puisque chaque distance apparait exactement une fois. Il est donc garantis que si l'on décale notre règle initiale d'un entier $k$, il n'existe qu'une graduation en commun entre ces 2 règles.

\vspace{1\baselineskip}

\textbf{2. Règles en 2 dimensions:} Une extension naturelle de la règle à mesurer est la règle en 2 dimensions, où les graduations sont placées sur un plan. La distance entre deux graduations est alors représentée sous forme de vecteur. Une règle en 2D parfaite permettrait de mesurer toutes les distances vectorielles possibles entre deux points sur une grille définie. Il est possible de définir une fonction $m_2(\ell)$ représentant le nombre minimal de graduations nécessaires pour mesurer toutes les distances vectorielles dans un carré de côté $\ell$. En adaptant les techniques utilisées pour les règles linéaires, on peut établir des bornes pour $m_2(\ell)$: $\sqrt{2} \cdot \ell \leq m_2(\ell) \leq 2\ell$. La borne supérieure peut être obtenue en divisant le carré initial par des carrés de longueur $\sqrt{\ell}$, et en plaçant $\ell$ graduations dans le carré en bas à gauche, puis dans la case supérieure droite de chaque autre carré.