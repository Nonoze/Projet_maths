\subsubsection*{(v) Déterminer les liens entre les versions rectiligne et circulaire du problème.}

Prenons maintenant une règle circulaire au lieu d'une règle rectiligne. Cela signifie que la règle est fermée sur elle-même, formant un cercle de longueur $\ell$. Le problème revient à pouvoir mesurer tous les arcs de cercle de longueur entière entre $1$ et $\ell$ à l'aide de graduations placées sur ce cercle. 

Définissons $m_{c}(\ell)$ comme le nombre minimal de graduations à placer.

La solution peut donc comprendre moins de graduations puisque la distance entre 2 graduations peut se faire dans les deux sens (sens horaire et antihoraire). Il faut donc seulement pouvoir mesurer des distances entre 1 et $\left\lfloor \frac{\ell}{2} \right\rfloor$. On obtient donc la relation suivante:
\[
m_{c}(\ell) \leq m( \left\lfloor \ell/2 \right\rfloor ) \leq 2\sqrt{ \left\lfloor \ell/2 \right\rfloor } \leq \sqrt{2\ell}.
\]

Nous pouvons aussi trouver une borne inférieure pour $m_{c}(\ell)$ en utilisant le même raisonnement que pour la règle rectiligne. En effet, avec $n$ graduations, on peut mesurer au plus $n(n-1)$ distances différentes . Il faut donc que:
\[n(n-1) \geq \ell .\]

Cela revient à dire que:

\[
m_{c}(\ell) \geq \frac{1 + \sqrt{1 + 4\ell}}{2} \sim \sqrt{\ell}.
\]

On encadre donc $m_{c}(\ell)$ de la façon suivante:
\[
\sqrt{\ell} \lesssim m_{c}(\ell) \lesssim \sqrt{2\ell}.
\]


Au niveau des solutions parfaites, il est possible de faire des généralisations.

\medskip

Tout d'abord, pour qu'une solution parfaite existe, il faut que la règle soit de longueur $\ell = q(q+1) + 1 = q^2 + q + 1$. En effet, il faut que la distance entre chaque paire de graduations soit unique. Pour un nombre $n = q+1$ de graduations, il y a donc $n(n-1) = q(q+1)$ paires, et donc $q(q+1)$ distances distinctes. La longueur vaut cette distance additionnée à 1: $\ell = q(q+1) + 1 = q^2 + q + 1$.

\medskip

Maintenant que nous avons nos critères, il faut déterminer pour quels $q$ il est possible de construire une solution parfaite.

\vspace{1\baselineskip}

\textbf{Proposition:} Une solution parfaite pour la règle circulaire de longueur $\ell = q^2 + q + 1$ avec $n = q + 1$ graduations existe lorsque $q$ est une puissance d'un nombre premier.

\vspace{2\baselineskip}

Solution attendue: $q = p^k$ où p est un nombre premier et k est un naturel non nul (séquence: \href{https://oeis.org/A335865}{OEIS A335865}).