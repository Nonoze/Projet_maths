\subsubsection*{(iv) Déterminer à partir de quel \( \ell \) qu'il n'existe pas de solution parfaite. Une solution parfaite est une configuration de graduations telle que chaque longueur de 1 à \( \ell \) soit mesurable d'une seule manière.}

\vspace{1\baselineskip}

Pour simplifier l'analyse, nous allons définir $n = m(\ell)$ (le nombre de graduations utilisées).

\medskip

Une solution est parfaite si et seulement si $ \ell = \binom{n}{2} $.  
Nous pouvons réécrire la règle en fonction de la distance entre les graduations adjacentes.  
Soit \( d_i \) la distance entre la graduation \( i \) et la graduation \( i+1 \) pour \( i = 1, 2, \ldots, n-1 \).  
Ansi, la somme de toutes ces distances vaut \( \ell \).

\medskip

Sous cette forme, la condition d'existence d'une solution parfaite revient à exiger que les sommes de tous les sous-tableaux contigus des distances \( d_i \) soient distinctes et couvrent exactement l'ensemble des longueurs de \( 1 \) à \( \ell \).  
Autrement dit, pour tout entier $k \in [1, \ell]$, il existe un unique couple \( (i, j) \) tel que  
\[
    k = d_i + d_{i+1} + \ldots + d_j .
\]

\medskip

Ce changement de perspective nous permet de raisonner en termes de sommes plutôt qu'en termes de différences.

\medskip

Une solution parfaite exige que tous les \( d_i \) soient des naturels distincts non nuls.  
Or, la plus petite somme de $n-1$ naturels distincts non nuls vaut  
\[
1 + 2 + \ldots + (n-1) = \frac{(n-1)n}{2} = \binom{n}{2} = \ell .
\]

\medskip

Le tableau $d$ est donc formé d'une permutation de $[1, 2, \ldots, n-1]$ telle que toutes les sommes de sous-tableaux contigus soient différentes.

\medskip

Montrons maintenant qu'il n'existe pas de solution parfaite pour \( n \geq 5 \). \\
Pour éviter qu'une longueur soit générée par des sous-tableaux différents, la valeur \(1\) ne peut être adjacente qu'à \(n-1\).  
On doit donc placer \(1\) à l'une des extrémités du tableau, disons \( d_1 = 1 \), ce qui impose \( d_2 = n-1 \).  
La valeur \(2\) doit également être adjacente à \(n-1\), sous peine de produire une longueur déjà réalisable.  
On fixe donc \( d_3 = 2 \).  
Aucune des valeurs restantes ne peut donc être placée en \( d_4 \) sans engendrer une somme déjà atteinte auparavant.  
Ainsi, la construction devient impossible pour \( n \geq 5 \).

\medskip

Nous avons donc démontré que la plus grande valeur de \( \ell \) pour laquelle une solution parfaite existe est \( n = 4 \) (et \( \ell = 6 \)).

\medskip

Les solutions parfaites sont résumées dans le tableau suivant:

\begin{center}
\begin{tabular}{|c|c|c|}
\hline
\( m(\ell) \) & \( \ell \) & Graduations \\
\hline
2 & 1 & \( \{0, 1\} \) \\
\hline
3 & 3 & \( \{0, 1, 3\} \) \\
\hline
\multirow{2}{*}{4} & \multirow{2}{*}{6} & \( \{0, 1, 4, 6\} \) \\
& & \( \{0, 2, 5, 6\} \) \\
\hline
\end{tabular}
\end{center}