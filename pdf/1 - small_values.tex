\subsubsection*{(i) Déterminer les solutions pour $ \ell \leq 213$.}

Voici quelques exemples de solutions pour $\ell$ allant de 1 à 10:

$m(1) = 2$: Nous avons besoin de 2 graduations à 0 et 1.

$m(2) = 3$: Nous avons besoin de 3 graduations à 0, 1 et 2.

$m(3) = 3$: Nous avons besoin de 3 graduations à 0, 1 et 3.

$m(4) = 4$: Nous avons besoin de 4 graduations à 0, 1, 2 et 4.

$m(5) = 4$: Nous avons besoin de 4 graduations à 0, 1, 2 et 5.

$m(6) = 4$: Nous avons besoin de 4 graduations à 0, 1, 4 et 6.

$m(7) = 5$: Nous avons besoin de 5 graduations à 0, 1, 2, 3 et 7.

$m(8) = 5$: Nous avons besoin de 5 graduations à 0, 1, 2, 5 et 8.

$m(9) = 5$: Nous avons besoin de 5 graduations à 0, 1, 2, 6 et 9.

$m(10) = 6$: Nous avons besoin de 6 graduations à 0, 1, 2, 3, 6 et 10.


\vspace{2\baselineskip}


Graphique des valeurs de $m(\ell)$ pour $\ell \leq 231$ :
\begin{center}
	\includegraphics[width=1\textwidth]{../data/graph_f.png}
\end{center}

\vspace{2\baselineskip}

On peut déjà que la fonction $m(\ell)$ évolue proportionnellement à $ \sqrt{\ell} $.