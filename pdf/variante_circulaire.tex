\underline{\textit{(iv) Quels liens peut-on établir entre les versions rectiligne et circulaire du problème?}}.

\vspace{1\baselineskip}

Considérons une règle circulaire de longueur $l$ avec $m(l)$ graduations permettant de mesurer toutes les longueurs entières de $1$ à $l$. La solution peut donc comprendre moins de graduations puisqu'une distance entre 2 graduations peut être mesurée dans les 2 sens (sens horaire et antihoraire). Il faut donc seulement pouvoir mesurer des distances entre 1 et $\lfloor \frac{l}{2} \rfloor$.
Ainsi, si l'on note $m_{c}(l)$ le nombre minimal de graduations pour une règle circulaire de longueur $l$, on a la relation suivante:
\[
m_{c}(l) \leq m( \lfloor l/2 \rfloor ) \leq 2\sqrt{ \lfloor l/2 \rfloor } \leq \sqrt{2l}.
\]