\subsubsection*{(iii) Trouver des bornes pour $m(\ell)$.}

1. \textbf{Borne inférieure:} $m(\ell) \geq \frac{1+\sqrt{8\ell+1}}{2}$.

\begin{itemize}[leftmargin=2cm]
	\item \textbf{Démonstration:} Pour tout $n \in \mathbb{N}$, le nombre de paires $(i,j)$ avec $1 \leq i < j \leq n$ est donné par $\binom{n}{2} = \frac{n(n-1)}{2}$. Si le nombre de graduations est $m(\ell)$, alors le nombre de différences distinctes que l'on peut obtenir est au plus $\binom{m(\ell)}{2}$. Pour mesurer toutes les longueurs de 1 à $\ell$, il faut donc que:
	\[
		\binom{m(\ell)}{2} \geq \ell
	\]
	Cela conduit à l'inégalité quadratique:
	\[
		\frac{m(m-1)}{2} \geq \ell \iff m^2 - m - 2\ell \geq 0
	\]
	En résolvant cette inégalité, on obtient:
	\[
		m(\ell) \geq \left\lceil \frac{1 + \sqrt{1 + 8\ell}}{2} \right\rceil
	\]
	Ceci établit la borne inférieure.

\end{itemize}




\vspace{1\baselineskip}


2. \textbf{Borne supérieure:} $m(\ell) \leq 2\sqrt{\ell} + 1$.

\begin{itemize}[leftmargin=2cm]
	\item \textbf{Démonstration:} Considérons une règle avec des graduations placées aux positions $0, 1, 2, \ldots, n-1$ et aux positions $2n - 1, 3n - 1, \ldots, kn - 1, \ell$ où $k = \lfloor \frac{\ell}{n} \rfloor$. Le nombre total de graduations est donc $n + k - 1 + 1 = n + k = n + \lfloor \frac{\ell}{n} \rfloor$.
	Pour mesurer toutes les longueurs de 1 à $\ell$, il suffit de choisir $n$ tel que $n \approx \sqrt{\ell}$. En effet, en choisissant $n = \left\lceil \sqrt{\ell} \right\rceil$, on a:
	\[
		m(\ell) \leq n + \left\lfloor \frac{\ell}{n} \right\rfloor \leq \left\lceil \sqrt{\ell} \right\rceil + \left\lfloor \frac{\ell}{\left\lceil \sqrt{\ell} \right\rceil} \right\rfloor \leq \left\lceil \sqrt{\ell} \right\rceil + \left\lfloor \sqrt{\ell} \right\rfloor \leq 2 \left\lfloor \sqrt{\ell} \right\rfloor + 1
	\]
	Ceci établit la borne supérieure.
\end{itemize}



\vspace{2\baselineskip}

Nous avons donc montré que:

\begin{align*}
	\frac{1 + \sqrt{1 + 8\ell}}{2} \leq m(\ell) \leq 2 \left\lfloor \sqrt{\ell} \right\rfloor + 1 \\
	\implies \frac{1 + \sqrt{8\ell}}{2} \leq m(\ell) \leq 2 \left\lfloor \sqrt{\ell} \right\rfloor + 1 \\
	\implies \frac{1}{2} + \frac{2\sqrt{2\ell}}{2} \leq m(\ell) \leq 2 \left\lfloor \sqrt{\ell} \right\rfloor + 1 \\
	\implies \frac{1}{2} + \sqrt{2\ell} \leq m(\ell) \leq 2 \left\lfloor \sqrt{\ell} \right\rfloor + 1 \\
	\implies \sqrt{2\ell} \leq m(\ell) \leq 2\sqrt{\ell} + 1
\end{align*}