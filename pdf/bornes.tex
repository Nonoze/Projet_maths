\underline{\textit{(ii) Trouver des bornes pour $m(l)$}}.

\vspace{1\baselineskip}

1. \textbf{Borne inférieure:} $m(l) \geq \frac{1+\sqrt{8l+1}}{2}$.

\begin{itemize}[leftmargin=2cm]
	\item \textbf{Démonstration:} Pour tout $n \in \mathbb{N}$, le nombre de paires $(i,j)$ avec $1 \leq i < j \leq n$ est donné par $\binom{n}{2} = \frac{n(n-1)}{2}$. Si le nombre de graduations est $m(l)$, alors le nombre de différences distinctes que l'on peut obtenir est au plus $\binom{m(l)}{2}$. Pour mesurer toutes les longueurs de 1 à $l$, il faut donc que:
	\[
		\binom{m(l)}{2} \geq l
	\]
	Cela conduit à l'inégalité quadratique:
	\[
		\frac{m(m-1)}{2} \geq l \iff m^2 - m - 2l \geq 0
	\]
	En résolvant cette inégalité, on obtient:
	\[
		m(l) \geq \frac{1 + \sqrt{1 + 8l}}{2}
	\]
	Ceci établit la borne inférieure.

\end{itemize}




\vspace{1\baselineskip}


2. \textbf{Borne supérieure:} $m(l) \leq 2\sqrt{l} + 1$.

\begin{itemize}[leftmargin=2cm]
	\item \textbf{Démonstration:} Considérons une règle avec des graduations placées aux positions $0, 1, 2, \ldots, n-1$ et aux positions $2n - 1, 3n - 1, \ldots, kn - 1, l$ où $k = \lfloor \frac{l}{n} \rfloor$. Le nombre total de graduations est donc $n + k - 1 + 1 = n + k = n + \lfloor \frac{l}{n} \rfloor$.
	Pour mesurer toutes les longueurs de 1 à $l$, il suffit de choisir $n$ tel que $n \approx \sqrt{l}$. En effet, en choisissant $n = \lceil \sqrt{l} \rceil$, on a:
	\[
		m(l) \leq n + \lfloor \frac{l}{n} \rfloor \leq \lceil \sqrt{l} \rceil + \lfloor \frac{l}{\lceil \sqrt{l} \rceil} \rfloor \leq \lceil \sqrt{l} \rceil + \lfloor \sqrt{l} \rfloor \leq 2 \lfloor \sqrt{l} \rfloor + 1
	\]
	Ceci établit la borne supérieure.
\end{itemize}



\vspace{2\baselineskip}

Nous avons donc montré que:

\begin{align*}
	\frac{1 + \sqrt{1 + 8l}}{2} \leq m(l) \leq 2 \lfloor \sqrt{l} \rfloor + 1 \\
	\implies \frac{1 + \sqrt{8l}}{2} \leq m(l) \leq 2 \lfloor \sqrt{l} \rfloor + 1 \\
	\implies \frac{1}{2} + \frac{2\sqrt{2l}}{2} \leq m(l) \leq 2 \lfloor \sqrt{l} \rfloor + 1 \\
	\implies \frac{1}{2} + \sqrt{2l} \leq m(l) \leq 2 \lfloor \sqrt{l} \rfloor + 1 \\
	\implies \sqrt{2l} \leq m(l) \leq 2\sqrt{l} + 1
\end{align*}